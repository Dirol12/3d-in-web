Для сборки проекта в пакет, пригодный для использования на пользовательском компьютере был выбран компоновщик Webpack\cite{webpack}\cite{webpack2}, помогающий тривиализировать сложную задачу
автоматизации построения релизного пакета приложения. Функционал Webpack упрощает процесс разработки одностраничных приложений, реализуя функционал продвинутого
разделения кода и <<горячей перезагрузки ресурсов>> (Hot Reload) для более быстрой разработки с помощью компонентной технологии пользовательского интерфейса, такой как React.
Webpack -- система сборки, которая предоставляет не только функционал компоновки модулей, но и может выполнять задачи, которые обычно выполняют специализированные task-runner системы,
такие как Gulp или Grunt. К тому же, возможности Webpack не ограничиваются обработкой JavaScript-файлов, так как он может работать с любыми видами статических ресурсов веб среды:
CSS (и языки, транслируемые в CSS), изображения, html-компоненты и, после реализации соответствующего модуля-загрузчика, любой другой тип содержимого. Webpack также поддерживает
полезную при разработки сложных веб-приложений функцию -- code splitting (разбиение кода) и dependency tree shaking (прочесывания дерева зависимостей). Большое приложение можно
разбить на подмодули и система упаковки автоматически включит их в скомпилированный ресурсный файл, в случае если он используется где-либо в системе. Все ненужные ресурсы не попадают
в результирующий исполняемый файл кроме случаев, когда это явно указано в конфигурации webpack (например, в случаях зависимости времени выполнения без соответствующей зависимости на
этапе компиляции).

Компоновщик Webpack позволяет упаковывать, компилировать, организовывать множество ресурсов и библиотек, необходимых для современного веб-проекта. Схема сборки веб проекта посредством
модулей-загруз-чиков представлена на рисунке \ref{figure:domain:webpack}.

\begin{figure}[ht]
\centering
  \includegraphics[scale=0.40]{webpack.png}
  \caption{Схема процесса сборки webpack}
  \label{figure:domain:webpack}
\end{figure}

Стандартные возможности Webpack включают, но не ограничиваются следующими:
\begin{itemize}
\item автоматическое построение дерева зависимостей ресурсов, в том числе зависимостей разных типов содержимого друг от друга (например, зависимости *.jsx или *.mjs файлов от каскадных таблиц
стилей *css или их прародителей, *.less/*.sass);
\item упаковка модулей, поддерживающих возможности ленивой (deferred, lazy loading) загрузки в отдельные файлы;
\item выполнение оптимизации кода, удаление ненужных элементов дерева зависимостей и минимизация получившихся исполняемых файлов;
\item поддержка функционала локального сервера ресурсов (dev server), позволяющего производить <<горячую перезагрузку ресурсов>> на странице прямо в процессе работы над проектом, что достигается
путем частичной перекомпиляции некоторых из ветвей зависимостей проекта.
\end{itemize}

Пример использования средств Webpack для организации автоматизированной сборки проекта:

Как и большинству инструментов Web-разработки, Webpack реализован на базе API Node.js \cite{wamp}\cite{node} -- серверной реализации JavaScript, построенной на базе движка V8. Установка Webpack производится с помощью
системы управления модулями node.js -- NPM \cite{npm} (Node package manager). Установка производится с помощью следующей команды в терминале:

\begin{lstlisting}[language=bash, label=lst:domain:html]
npm install webpack --global
\end{lstlisting}

Данная команда установит Webpack глобально в системе (по умолчанию установив его в директорию /usr/local/bin/npm/node\_modules), создав символические ссылки на необходимые исполняемые файлы, что
позволит запускать его из любой директории на компьютере разработчика. Далее, внутри директории проекта, был создан файл index.html с начальной разметкой:

\begin{lstlisting}[language=HTML, label=lst:domain:html]
<html lang="ru">
<head>
  <meta charset="UTF-8">
</head>
<body>
  <h2></h2>
  <script src="bundle.js"></script>
</body>
</html>
\end{lstlisting}

Важной частью этого кода является ссылка на файл bundle.js, который содержит в себе результат работы Webpack.
Первый файл определяет начальную точку приложения, в которой Webpack будет искать все зависимости. Это сработает и в том случае, если в вызываемых зависимостях 
есть свои зависимости от других модулей -- до тех пор, пока не подключатся абсолютно все необходимые модули. Таким образом, на выход получится один файл bundle.js со всем модулями.
В проекте была собрана конфигурация webpack с React.js

Точка входа в приложение:
\begin{lstlisting}[language=TypeScript, label=lst:domain:html]
const config = {
  entry: {
    app: './src/js/app.js'
  }
\end{lstlisting}

Параметры финальной упаковки:
\begin{lstlisting}[language=TypeScript, label=lst:domain:html]
  output: {
    filename: 'bundle.js',
    path: distPath
  }
\end{lstlisting}

Одной из самых важных особенностей Webpack, является возможность использовать loader. Loader по сути своей являются аналогами “задач” (tasks) в Grunt и Gulp. По существу,
они принимают содержимое файлов, а затем преобразуют его необходимым образом и включают результат преобразования в общую сборку.
Подключение React.js в бандлере:
\begin{lstlisting}[language=TypeScript, label=lst:domain:html]
 module: {
    rules: [{
      test: /\.js$/,
      exclude: [/node_modules/],
      use: [{
        loader: 'babel-loader',
        options: {
          presets: ['env', 'react']
        }
      }]
    }.
\end{lstlisting}

Для увелечения гибкости стиля был выбран язык SASS\cite{sass}\cite{sass2}\cite{sass3}. SASS это язык похожий на HAML (весьма лаконичный шаблонизатор), но предназначенный 
для упрощения создания CSS-кода. Проще говоря, SASS это такой язык, код которого специальной ruby-программой транслируется в обычный CSS код. Синтаксис этого языка очень гибок, 
он учитывает множество мелочей, которые так желанны в CSS. 
Изменение файла конфигурации для подключение подгрузки стилей:

\begin{lstlisting}[language=TypeScript, label=lst:domain:html]
test: /\.scss$/,
      exclude: [/node_modules/],
      use: extractSass.extract({
        fallback: 'style-loader',
        use: [{
          loader: 'css-loader',
          options: {
            modules: true,
            sourceMap: true,
            importLoaders: 2,
            localIdentName: '[name]__[local]__[hash:base64:5]', // className template
            minimize: isProduction
          }
        },
          'sass-loader',
          'resolve-url-loader'
        ]
      })
\end{lstlisting}

Для работы с изображениями мы будем использовать url-loader, который является ещё одним loader для Webpack. Он берёт относительные URL 
ваших изображений и изменяет их таким образом, чтобы они корректно подключались в общем файле.
\begin{lstlisting}[language=TypeScript, label=lst:domain:html]
test: /\.(gif|png|jpe?g|svg)$/i,
      use: [{
        loader: 'file-loader',
        options: {
          name: 'images/[name][hash].[ext]'
        }
      }, {
        loader: 'image-webpack-loader',
        options: {
          mozjpeg: {
            progressive: true,
            quality: 70
          }
        }
      },
      ],}, {
      test: /\.(eot|svg|ttf|woff|woff2)$/,
      use: {
        loader: 'file-loader',
        options: {
          name: 'fonts/[name][hash].[ext]'
        }
      },
    }]
  }
\end{lstlisting}
