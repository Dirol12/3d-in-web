В контексте трехмерной компьютерной графики, 3D-моделированием (либо трехмерным моделированием) называют процесс разработки
математического представления трехмерной поверхности объектов (живых либо неживых по своей природе) с использованием специальных
инструментов. Продукт трехмерного моделирования -- 3D-модель объекта. Результирующая трехмерная модель может быть отображена
на двухмерном носителе либо напечатана с помощью 3D-принтера либо использована в компьютерной симуляции.

Модели могут быть созданы в автоматическом режеми либо вручную. Процесс ручного создания трехмерной модели по своей сущности
схож с процессом создания скульптуры.

Программное обеспечение трехмерного моделирования, известное под названием 3D-модельер - категория графических приложений, использующихся для создания трехмерных
моделей.

% - Представление моделей
Практически все трехмерные модели могут быть условно поделены на две основные категории.
Цельные -- данные модели определяют объем объекта, который они представляют. Цельные модели в основном используются в инжеренрых и медицинских симуляциях
и обычно строятся с помощью геометрических примитивов.\cite{solid_modelling}

Оболочечные либо поверхностные -- данные модели представляют поверхность объекта (например, сферу шара, либо бесконечно тонкую яичную скорлупу).
Основная область применения таких моделей -- игровая и киноиндустрия, визуализация данных.

Цельные и поверхностные модели функциональ представляют идентичные объекты. Разница между ними в основном состоит в том, каким образом они
определяются и редактируются, а так же набор соглашений используемых при работе с такими моделями, например -- правила и типы аппроксимации
при вычислениях. 

Поверхностные модели должны быть непрерывными (то есть не иметь отверстий, трещин в оболочке) для того чтобы использовать их в
качестве натурального объекта. Полигональные сетки (и, в несколько меньшей степени, подразделенные поверхности) являются наиболее широко распространенной
категорией поверхностных моделей. Горизонтально выровненные наборы полигонов являются полезным инструментов в представлении видоизменяющихся или
деформирующихся поверхностей, подверженных большому числу топологических изменений, таких как, например, поверхность жидкости.

Процесс трансформации представления объектов, например, таких как координаты цетральной точки сферы и ее радиуса, в полигональное представление называется тесселяцией.
Этот процесс генерирует полигональный рендеринг, в котором объекты представленные в виде абстрактных фигур (называемые примитивами - сферы, конусы и т.п) 
преобразуются в полигональные меши, состоящие из сети взаимосвязынных треугольников. 

Меши, существующие в виде треугольников, гораздо популярнее, нежели меши основанные из квадратах, 
потому что было доказано, что они легче поддаются растеризации (поверхность, описываемая каждым треугольником, является плоской, поэтому проекция всегда выпукла).
Полигональное представление не всегда используется в процессе рендеринга, и в этих случаях тесселяция не включена в преобразование из абстрактного представления в конечный результат рендеринга.

% - Процесс моделирования
Существует три популярных пути представления модели:

1) Полигональное моделирование 
Вершины в трехмерном пространстве, называемые вертексами, соединенные отрезками прямых, формирует полигональную сеть.
Подавляющее большинство существующих трехмерных моделей на сегодня основано на текстурированных полигональных моделях, из-за их гибкости
и возможности компьютеров использовать сплайновое моделирование.
%Polygonal modeling – Points in 3D space, called vertices, are connected by line segments to form a polygon mesh. The vast
%majority of 3D models today are built as textured polygonal models, because they are flexible and because computers can
%rurve modeling – Surfaces are defined by curves, which are influenced by weighted control points. The curve follows (but
%does not necessarily interpolate) the points. Increasing the weight for a point will pull the curve closer to that point.
%Curve types include nonuniform rational B-spline (NURBS), splines, patches, and geometric primitives
%Digital sculpting – Still a fairly new method of modeling, 3D sculpting has become very popular in the few years it has
%been around.[citation needed] There are currently three types of digital sculpting: Displacement, which is the most widely
%sed among applications at this moment, uses a dense model (often generated by subdivision surfaces of a polygon control
%mesh) and stores new locations for the vertex positions through use of an image map that stores the adjusted locations.
%Volumetric, loosely based on voxels, has similar capabilities as displacement but does not suffer from polygon stretching
%when there are not enough polygons in a region to achieve a deformation. Dynamic tessellation is similar to voxel but divides
%the surface using triangulation to maintain a smooth surface and allow finer details. These methods allow for very artistic
%exploration as the model will have a new topology created over it once the models form and possibly details have been sculpted.
%The new mesh will usually have the original high resolution mesh information transferred into displacement data or normal map
%data if for a game engine.

% References
% "3D Scanning Advancements in Medical Science". Konica Minolta. Retrieved 24 October 2011.
% Jon Radoff, Anatomy of an MMORPG Archived 2009-12-13 at the Wayback Machine., August 22, 2008
% "Lands' End First With New 'My Virtual Model' Technology: Takes Guesswork Out of Web Shopping for Clothes That Fit".
% PRNewswire. Lands' End. February 12, 2004. Retrieved 2013-11-24.
% "All About Virtual Fashion and the Creation of 3D Clothing". CGElves. Retrieved 25 December 2015.
% "3D Clothes made for The Hobbit using Marvelous Designer". 3DArtist. Retrieved 9 May 2013.
% "What is 3D Printing? The definitive guide". 3D Hubs. Retrieved 2017-11-18.
% "3D Printing Toys". Business Insider. Retrieved 25 January 2015.
% "Printout3D—Wolfram Language Documentation". reference.wolfram.com. Retrieved 2016-08-06.
% "New Trends in 3D Printing – Customized Medical Devices". Envisiontec. Retrieved 25 January 2015.
 %Sikos, L. F. (2016). Rich Semantics for Interactive 3D Models of Cultural Artifacts. Communications in Computer and
 %Information Science. 672. Springer International Publishing. pp. 169–180. doi:10.1007/978-3-319-49157-8_14.
 %Yu, D.; Hunter, J. (2014). "X3D Fragment Identifiers—Extending the Open Annotation Model to Support Semantic Annotation
 %of 3D Cultural Heritage Objects over the Web". International Journal of Heritage in the Digital Era. 3 (3): 579–596.
 %doi:10.1260/2047-4970.3.3.579.