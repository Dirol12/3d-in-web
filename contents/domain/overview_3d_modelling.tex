В контексте трехмерной компьютерной графики, 3D-моделированием (либо трехмерным моделированием) называют процесс разработки
математического представления трехмерной поверхности объектов (живых либо неживых по своей природе) с использованием специальных
инструментов. Продукт трехмерного моделирования -- 3D-модель объекта. Результирующая трехмерная модель может быть отображена
на двухмерном носителе либо напечатана с помощью 3D-принтера либо использована в компьютерной симуляции.

Модели могут быть созданы в автоматическом режеми либо вручную. Процесс ручного создания трехмерной модели по своей сущности
схож с процессом создания скульптуры.

Программное обеспечение трехмерного моделирования, известное под названием 3D-модельер - категория графических приложений, использующихся для создания трехмерных
моделей.

% - Представление моделей
Almost all 3D models can be divided into two categories.

Solid – These models define the volume of the object they represent (like a rock). Solid models are mostly used for engineering
and medical simulations, and are usually built with constructive solid geometry
Shell/boundary – these models represent the surface, e.g. the boundary of the object, not its volume
(like an infinitesimally
thin eggshell). Almost all visual models used in games and film are shell models.
Solid and shell modeling can create functionally identical objects. Differences between them are mostly variations in the way
they are created and edited and conventions of use in various fields and differences in types of approximations between the
model and reality.

Shell models must be manifold (having no holes or cracks in the shell) to be meaningful as a real object. Polygonal meshes
(and to a lesser extent subdivision surfaces) are by far the most common representation. Level sets are a useful representation
for deforming surfaces which undergo many topological changes such as fluids.

The process of transforming representations of objects, such as the middle point coordinate of a sphere and a point on its
circumference into a polygon representation of a sphere, is called tessellation. This step is used in polygon-based rendering,
where objects are broken down from abstract representations ("primitives") such as spheres, cones etc., to so-called meshes,
which are nets of interconnected triangles. Meshes of triangles (instead of e.g. squares) are popular as they have proven to
be easy to rasterise (the surface described by each triangle is planar, so the projection is always convex); .[4] Polygon 
representations are not used in all rendering techniques, and in these cases the tessellation step is not included in the
transition from abstract representation to rendered scene.

% - Процесс моделирования
There are three popular ways to represent a model:

Polygonal modeling – Points in 3D space, called vertices, are connected by line segments to form a polygon mesh. The vast
majority of 3D models today are built as textured polygonal models, because they are flexible and because computers can
render them so quickly. However, polygons are planar and can only approximate curved surfaces using many polygons.
Curve modeling – Surfaces are defined by curves, which are influenced by weighted control points. The curve follows (but
does not necessarily interpolate) the points. Increasing the weight for a point will pull the curve closer to that point.
Curve types include nonuniform rational B-spline (NURBS), splines, patches, and geometric primitives
Digital sculpting – Still a fairly new method of modeling, 3D sculpting has become very popular in the few years it has
been around.[citation needed] There are currently three types of digital sculpting: Displacement, which is the most widely
used among applications at this moment, uses a dense model (often generated by subdivision surfaces of a polygon control
mesh) and stores new locations for the vertex positions through use of an image map that stores the adjusted locations.
Volumetric, loosely based on voxels, has similar capabilities as displacement but does not suffer from polygon stretching
when there are not enough polygons in a region to achieve a deformation. Dynamic tessellation is similar to voxel but divides
the surface using triangulation to maintain a smooth surface and allow finer details. These methods allow for very artistic
exploration as the model will have a new topology created over it once the models form and possibly details have been sculpted.
The new mesh will usually have the original high resolution mesh information transferred into displacement data or normal map
data if for a game engine.

% References
 "ERIS Project Starts". ESO Announcement. Retrieved 14 June 2013.
 "What is Solid Modeling? 3D CAD Software. Applications of Solid Modeling". Brighthub Engineering. Retrieved 2017-11-18.
 "3D Scanning Advancements in Medical Science". Konica Minolta. Retrieved 24 October 2011.
 Jon Radoff, Anatomy of an MMORPG Archived 2009-12-13 at the Wayback Machine., August 22, 2008
 "Lands' End First With New 'My Virtual Model' Technology: Takes Guesswork Out of Web Shopping for Clothes That Fit".
 PRNewswire. Lands' End. February 12, 2004. Retrieved 2013-11-24.
 "All About Virtual Fashion and the Creation of 3D Clothing". CGElves. Retrieved 25 December 2015.
 "3D Clothes made for The Hobbit using Marvelous Designer". 3DArtist. Retrieved 9 May 2013.
 "What is 3D Printing? The definitive guide". 3D Hubs. Retrieved 2017-11-18.
 "3D Printing Toys". Business Insider. Retrieved 25 January 2015.
 "Printout3D—Wolfram Language Documentation". reference.wolfram.com. Retrieved 2016-08-06.
 "New Trends in 3D Printing – Customized Medical Devices". Envisiontec. Retrieved 25 January 2015.
 Sikos, L. F. (2016). Rich Semantics for Interactive 3D Models of Cultural Artifacts. Communications in Computer and
 Information Science. 672. Springer International Publishing. pp. 169–180. doi:10.1007/978-3-319-49157-8_14.
 Yu, D.; Hunter, J. (2014). "X3D Fragment Identifiers—Extending the Open Annotation Model to Support Semantic Annotation
 of 3D Cultural Heritage Objects over the Web". International Journal of Heritage in the Digital Era. 3 (3): 579–596.
 doi:10.1260/2047-4970.3.3.579.