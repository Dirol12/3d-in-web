Всемирная паутина (WWW или для простоты понимания Веб) -- это глобальная информационная сеть, в которой пользователи могут что-либо писать или читать, если у них есть доступ к компьютеру, подсоединенного к Интернету.
Это понятие часто ошибочно используют как синоним к слову Интернет, но веб это сервис, который работает через Интернет, как и обычная электронная почта. История Интернета зарождается гораздо раньше, нежели история World Wide Web.
Веб -- это глобальная информационная система.

Функция веб как прикладного протокола -- работа над системой Интернет, реализующая прикладной функционал Интернета, увеличивая его полезность для повседневных пользователей.
Развитие веб браузеров улучшило качество работы с сетью, позволило отобразить изображения и анимации, сложные документы. Термины <<Интернет>> и <<Всемирная паутина>> считаются взаимозаменяемыми, однако
в действительности они означают разные вещи. Интернет -- глобальная система взаимосоединенных компьютерных сетей, в то время как Веб -- глобальная коллекция документов и прочих ресурсов, соединенных гиперссылками и
уникальными ресурсными идентификаторами (URI). Ресурсы веб обычно доступны с помощью протокола HTTP (HTTPS), который является одним из самых распространенных протоколов интернет-коммуникации. [1]

Отображение страницы в веб обычно начинается с набора ее уникального ресурсного идентификатора в адресную строку веб браузера, либо с открытия гиперссылки, указывающей на эту страницу.
Веб браузер в дальнейшем начинает серию фоновых сессий коммуникаций для того чтобы загрузить и отобразить запрошенный ресурс.

Процесс обработки запроса можно описать следующим алгоритмом:
\begin{enumerate}[label=\arabic*.]
\item Веб-браузер разрешает имя сервера, указанное в URI для получения его адреса в протоколе интернет, используя глобальную распределенную систему доменных имен (DNS).
Процесс разрешения возвращает адрес вида 203.0.113.4 or 2001:db8:2e::7334.
\item{Браузер устанавливает TCP-соединение с сервером на порте 80, стандартном порте HTTP протокола и передает текст HTTP запроса. Запрос может иметь следующий вид:
\begin{lstlisting}[language=HTTP_HEADERS, label=lst:domain:http-request]
GET /home.html HTTP/1.1
Host: www.example.org
\end{lstlisting}
}
\item{Сервер, получив запрос, генерирует HTTP ответ, имеющий следующий вид: 
\begin{lstlisting}[language=HTTP_HEADERS, label=lst:domain:http-response]
HTTP/1.0 200 OK
Content-Type: text/html; charset=UTF-8
\end{lstlisting}
}
\item{Сервер, сгенерировав HTTP ответ, продолжает транслировать содержимое страницы в секцию Response. Содержимое страницы может иметь следующий вид: 
\begin{lstlisting}[language=HTML, label=lst:domain:html]
<html>
  <head>
    <title>Example.org - The World Wide Web</title>
  </head>
  <body>
    <p>
        The World Wide Web, abbreviated as
        WWW and commonly known ...
    </p>
  </body>
</html>
\end{lstlisting}
}
\item{Веб браузер производит структурный анализ полученной разметки и интерпретирует html-тэги (например, <title>, <p>) для того чтобы отформатировать текст на экране.
Многие веб-страницы используют средства HTML для того чтобы создать сссылки на существующие веб-ресурсы, такие как изображения, скрипты, определяющие поведение страницы и
каскадные таблицы стилей, которые определяют внешний вид отформатированной страницы. Браузеры могут делать дополнительные HTTP запросы к веб-серверу для получения этих
ресурсов. По мере того, как содержимое их передается с сервера на клиент, браузер инкрементально отображает и стилизует содержимое страницы в соответствии с тем, что указано
в тексте ее HTML разметки.
}
\end{enumerate}

%  [1] "What is the difference between the Web and the Internet?". World Wide Web Consortium. Archived from the original on 22 April 2016. Retrieved 18 April 2016.