Актуальность темы исследования. 
Сотни тысяч самых разных компаний в мире используют технологии объемного компьютерного моделирования для проектирования, дизайна и производства объектов любого уровня сложности: от упаковки 
газированных напитков до новейших самолетов. Причем для их создания используется одно и то же ПО, позволяющее создавать трехмерные цифровые макеты будущих предметов и процессов.
Раньше инженеры обходились двухмерными чертежами, но они не способны в полной мере передать устройство таких сложных механизмов как, скажем, летательные аппараты. Более того, для 
каждого ракурса требуется отдельный чертеж, а значит и отдельный лист бумаги.


Трехмерные цифровые макеты лишены недостатков бумажных чертежей: они просматриваемы под любым углом, позволяют рассмотреть каждую мельчайшую деталь конструируемого объекта и по-настоящему 
мобильны: цифровые прототипы отображаются практически на любых современных устройствах, и не требуют распечатки на бумаге. 

И, несмотря на то, что технологии трехмерной графики, используемые в десктопных приложениях, довольно долгое время шагают семимильными шагами, и за прошедшие 10 лет успело смениться 
несколько поколений 3D API и графического обеспечения, веб-разработчики не имели родной для веба технологии, позволяющей работать с 3D.
Ситуация начала меняться в другую сторону, когда Apple предложила API для работы с 3D в Web. Начинание Apple поддержали Google и Mozilla, и, несколько позднее, Opera. Так родилась 
спецификация WebGL для работы с 3D в веб-среде.

Удобство использования веб-браузера в качестве платформы для реализации приложения состоит в том, что работа приложения не зависит от особенностей платформы, на которой оно выполняется 
(за исключением частных особенностей различных интерпретаторов JavaScript), что позволяет переиспользовать код приложения на разных операционных системах (Будь то MS Windows, любые из 
дистрибутивов Linux, Google Web OS, OS X, и даже ОС мобильных устройств, например Android) реализовав минимальный набор polyfill функций. Еще один критерий, добавляющий удобства подобным 
web-based системам состоит в том, что очень многие приложения (Редакторы документов, чаты, медиаплееры) адаптируются к такой модели работы, что упростит взаимодействие с пользователем --
процесс его работы с приложением уже будет ему знаком по аналогии с другими приложениями. 
 
Также использование веб-браузера позволяет упростить процесс обновления приложения, так как оно 
находится в централизованной локации и доступ к нему управляется сервером, который контролируется разработчиками.

Анализ актуальности обусловили выбор темы исследования: «Возможности использования трехмерного моделирования в веб-проектировании и веб-дизайне»

Гипотеза исследования: Применение современных технологий Web-про-граммирования существенно улучшит процесс трехмерного моделирования, повысит удобство использования, позволит распределить 
нагрузку на вычислительные ресурсы.

Целью исследования является анализ и обзор инструментов для трехмерного моделирования в веб-среде, а также описание разработки Web-приложения.

Для достижения поставленной цели необходимо решить следующие задачи:  
\begin{itemize}
\item анализ существующих решений в области трехмерного моделирования
\item выбор оптимальных средств разработки с учетом существующих критериев  
\item разработка Web-приложения для трехмерного моделирования  
\item оценка пути дальнейшей оптимизации потребляемых ресурсов и пути дальнейшего развития проекта
\end{itemize}

Предметом исследования является веб-приложение для трехмерного моделирования в веб-среде.

Научная новизна и теоретическая значимость исследования. Работа открывает направление исследований в области развития современных информационных и Web-технологий, применения информационных 
и Web-технологий в трехмерном моделировании. Выявлены, обоснованы и описаны преимущества определенных информационных технологий как инструмента 
для отображения и создания трехмерных моделей. 

