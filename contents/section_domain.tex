\section{Обзор литературы}
\label{sec:domain}

\subsection{Обзор основных понятий 3D графики}
\label{sub:domain:overview_3d}
3D компьютерная графика -- способ представления геометрической информации в трехмерном пространстве (зачастую представленной
в Декартовой системе координат), сохраненной на компьютере с целью произведения вычислений и создания двухмерных изображений.
Результирующие изображения могут быть в сохранены для просмотра в будущем, либо отображаться в реальном времени. 

Процесс отображения 3D графики основан на многих алгоритмах работы с двухмерными векторными изображениями в стадии обработки
каркасов моделей (также называемых wireframe-моделями) и алгоритмах работы с растровой графикой при получении финального изображения.
Зачастую, из-за смешанного подхода к работе с трехмерной графикой, сложно выделить набор явных различий между 3D и 2D, так как многие
среды, предназначенные для работы с двухмерной графикой используют техники обработки трехмерных изображений (например, для достижения
реалистичной модели освещения), а системы работы с трехмерной графикой подразумевают использования техник обработки двухмерных изображений
(например, постобработка).

3D computer graphics are often referred to as 3D models. Apart from the rendered graphic, the model is contained within
the graphical data file. However, there are differences: a 3D model is the mathematical representation of any three-dimensional
object. A model is not technically a graphic until it is displayed. A model can be displayed visually as a two-dimensional image
through a process called 3D rendering or used in non-graphical computer simulations and calculations. With 3D printing, 3D models are similarly rendered into a 3D physical representation of the model, with limitations to how accurate the rendering can match the virtual model

Понятие объектов трехмерной графики часто отождествляют с понятием 3D моделей, однако эти понятия не совсем тождественны, так как 


- Что такое 3D графика
- Обзор понятий 3д моделирования
- Что такое веб
- Где ее используют
- Как ее можно применить в веб среде
- Целесообразно ли это
- Как можно перенести процесс создания 3д графики в веб
- Обзор подходов и технологий 3д графики в веб среде