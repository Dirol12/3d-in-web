Основной целью данного проекта является разработка программного продукта для интеграции трехмерной модели в веб-среду,
манипуляции трехмерной геометрией и загрузкой/выгрузкой трехмерной модели в файл.

Основные требования к разрабатываемой системе:
\begin{itemize}
\item Возможность отображения трехмерной модели в веб-документе
\item Возможность манипуляции трехмерной моделью в трехмерном пространстве (применение трех фундаментальных операций -- трансляция, поворот и масштабирование) 
\item Возможность загрузки трехмерной модели из файла стандартизированного формата (*.obj, *.blend)
\item Возможность выгрузки трехмерной модели в файл для последующего открытия (*.obj)
\end{itemize}

В качестве платформы для разработки
приложения избран веб-фрейм-ворк Reactjs\cite{react3} ввиду высокой совместимости его компонентной модели с полностью функциональной
парадигмой разработки - Redux, которая хорошо подходит для реализации приложения для редактирования пользовательского
контента.

За основной механизм отображения принята комбинация WebGL + HTML5 canvas, позволяющая интегрировать трехмерную графику в
веб-документ.

Для реализации вышеуказанного функционала не требуется централизованного сервиса или хранилища, поэтому приложение может быть
размещено в среде статического HTML веб-сайта, например на github.io pages. Для добавления коммерческого функционала и расширения и
улучшения существующего допустима реализация централизованного веб-сервера или кластера серверов для обработки высокопроизводительных
задач, таких как рендеринг сложных сцен в высоком разрешении (в условиях, когда ресурсы клиентского компьютера ограничены).

В качестве основного языка разработки приложения используется язык TypeScript, потому как он дает разработчикам возможность писать строготипизированный код,
проверяемый на этапе компиляции, что уменьшает необходимость покрытия некоторого тривиального кода системы модульными тестами для обеспечения типовой безопасности
во время выполнения приложения.