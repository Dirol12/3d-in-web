\input{preamble}
\pagenumbering{gobble}

\begin{document}
  \begin{center}
    \MakeUppercase{\textbf{Рецензия}}\\[1em]

    научного руководителя \\
    на дисссертацию на соискание степени магистра \\
    по специальности 1 - 36 80 08 <<Инженерная геометрия и компьютерная графика>> \\[1em]

    \textbf{Волк Александры Олеговны} \\[1em]

    {на тему: <<Возможности использования трехмерного моделирования в веб-проектировании и веб-дизайне>>}\\[1em]
  \end{center}

  Представленный на рецензию проект выполнен на 2 листах графического материала и на 65 печатных страницах пояснительной записки.
  Тема проекта является актуальной ввиду современных тенденций развития веб-технологий. Были проанализированы возможности по отображению трехмерной графики
  в веб-браузерах, рассмотрена целесообразность переноса инструментария по разработе трехмерных моделей в веб-среду, реализован инструмент с базовым функционалом
  для создания и редактирования трехмерных моделей на основе веб-приложения, что подтверждает практическую значимость проекта.

  По качеству исполнения графического материала, по объему и сложности проект удовлетворяет квалификационным требованиям к магистерской диссертации.

  К замечаниям по проекту следует отнести:
  \begin{itemize}
    \item не обоснован выбор использованного веб-фреймворка для создания приложения обертки для редактора;
    \item использование англоицизмов и неосвященных аббревиатур без их расшифровки.
  \end{itemize}

  В целом работа выполнена технически грамотно, в полном соответствии с техническим заданием и заслуживает оценки восемь баллов, а сама Волк А.О. -- присвоения ей степени магистра технических наук.
  
  \vfill
  \begin{center}
    \begin{tabular}{ p{0.65\textwidth}p{0.25\textwidth} }
      Рецензент \\
      к.т.н., доцент \\
      старший преподаватель \\
      кафедры ИГ БГТУ & А.Л. Калтыгин\\[1em]
    \end{tabular}

  \end{center}
\end{document}
